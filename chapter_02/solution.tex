\documentclass[a4paper]{book}

\usepackage{afterpage}
\usepackage[hypcap=false]{caption}
\usepackage{enumitem}	% 定制enumerate标号
\usepackage{geometry}
\geometry{%
	left=2cm,%
	right=2cm,%
	top=2cm,%
	bottom=2cm,%
	bindingoffset=0cm
}
\usepackage{hyperref}
\hypersetup{
    colorlinks=true,            %链接颜色
    linkcolor=blue,             %内部链接
    filecolor=magenta,          %本地文档
    urlcolor=cyan,              %网址链接
    pdftitle={Overleaf Example},
    pdfpagemode=FullScreen,
}
\usepackage[none]{hyphenat}	% 阻止长单词分在两行
\usepackage{longtable}
\usepackage{mathrsfs}	% 提供\mathscr字体
\usepackage[version=4]{mhchem}
\usepackage{multirow}
\usepackage{subcaption}

\RequirePackage[many]{tcolorbox}
\tcbset{
    boxed title style={colback=magenta},
	breakable,
	enhanced,
	sharp corners,
	attach boxed title to top left={yshift=-\tcboxedtitleheight,  yshifttext=-.75\baselineskip},
	boxed title style={boxsep=1pt,sharp corners},
    fonttitle=\bfseries\sffamily,
}

\definecolor{skyblue}{rgb}{0.54, 0.81, 0.94}

\newtcolorbox[auto counter, number within=chapter, number format=\arabic]{exercise}[1][]{
    title={Exercise~\thetcbcounter},
    colframe=skyblue,
    colback=skyblue!12!white,
    boxed title style={colback=skyblue},
    overlay unbroken and first={
        \node[below right,font=\small,color=skyblue,text width=.8\linewidth]
        at (title.north east) {#1};
    }
}

\newtcolorbox[auto counter, number within=chapter, number format=\arabic]{solution}[1][]{
%    top=2ex,
%    boxrule=0pt,
%    leftrule=1.4pt,
    title={Solution~\thetcbcounter},
    colframe=teal!60!green,
    colback=green!12!white,
    boxed title style={colback=teal!60!green},
    overlay unbroken and first={
        \node[below right,font=\small,color=red,text width=.8\linewidth]
        at (title.north east) {#1};
    }
}

\newcommand{\AO}{{\rm AO}}
\newcommand{\Heff}{H^{\rm eff,\pi}}
\newcommand{\Hp}{H^\prime}
\newcommand{\Sp}{S^\prime}
\newcommand{\RRR}{{\rm R}^3}
\newcommand\Figref[1]{Fig \ref{#1}}
\newcommand\Tableref[1]{Table \ref{#1}}
\newcommand{\orb}[1]{{\rm #1}}
\newcommand{\orbp}{\orb{p}}

\allowdisplaybreaks

\begin{document}

	\setcounter{chapter}{2}

	\begin{exercise}
		Give all the symmetry elements of $\ce{H2O}$, $\ce{NH3}$ and $\ce{CH4}$. For each molecule list the symmetry operations which commute.
	\end{exercise}

	\begin{solution}
		\url{https://symotter.org/gallery}
		\begin{enumerate}[label=(\alph*)]
		
		\item $\ce{H2O}$ belongs to the point group $\mathscr{C}_{\rm 2v}$, which has 4 symmetry elements, viz. $E$, $C_2$, $\sigma_{\rm v}(xz)$, $\sigma_{\rm v}(yz)$.
		
		\item $\ce{NH3}$ belongs to the point group $\mathscr{C}_{\rm 3v}$, which has 6 symmetry elements, viz. $E$, $C_3$, $C^2_3$, $\sigma_{\rm v1}$, $\sigma_{\rm v2}$, $\sigma_{\rm v3}$.
		
		\item $\ce{CH4}$ belongs to the point group $\mathscr{T}_{\rm d}$, which has 24 symmetry elements, viz. $E$, 
		
		\end{enumerate}
		
	\end{solution}
	
	\begin{exercise}
		On the basis of symmetry, which of the following molecules cannot have a dipole moment: $\ce{CH4}$, $\ce{CH3Cl}$, $\ce{CH2D2}$, $\ce{H2S}$, $\ce{SF6}$?
	\end{exercise}

	\begin{solution}
	
		\begin{enumerate}[label=(\alph*)]
		
		\item $\ce{CH4}$ has no dipole moment.
		
		\item $\ce{CH3Cl}$ has a dipole moment.
		
		\item $\ce{CH2D2}$ has a dipole moment.
		
		\item $\ce{H2S}$ has a dipole moment.
		
		\item $\ce{SF6}$ has no dipole moment.
		
		\end{enumerate}

	\end{solution}

	\begin{exercise}
	Which of the following molecules cannot be optically active: $\ce{CHFClBr}$, $\ce{H2O2}$, $\ce{[Co(en)3]^{3+}}$, {\it cis}-$\ce{[Co(en)2}$ $\ce{(NH3)2]^{3+}}$, {\it trans}-$\ce{[Co(en)2(NH3)2]^{3+}}$?
	\end{exercise}
	
	\begin{solution}
		
		\begin{enumerate}[label=(\alph*)]
		
		\item $\ce{CHFClBr}$ belongs to the point group $\mathscr{C}_{\rm 1}$, which is optically active.
		
		\item $\ce{H2O2}$ belongs to the point group $\mathscr{C}_{\rm 2}$, which is optically active.
		
		\item $\ce{[Co(en)3]^{3+}}$ belongs to the point group $\mathscr{D}_{\rm 3}$, which is optically active.
		
		\item {\it cis}-$\ce{[Co(en)2(NH3)2]^{3+}}$ belongs to the point group $\mathscr{C}_{\rm 2}$, which is optically active.
		
		\item {\it trans}-$\ce{[Co(en)2(NH3)2]^{3+}}$ belongs to the point group $\mathscr{C}_{\rm 2h}$, which is optically inactive.
		
		\end{enumerate}
		
	\end{solution}

\end{document}