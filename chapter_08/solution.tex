\documentclass[a4paper]{book}

\usepackage{geometry}
\geometry{%
	left=2cm,%
	right=2cm,%
	top=2cm,%
	bottom=2cm,%
	bindingoffset=0cm}
\usepackage{enumitem}	% 定制enumerate标号
\usepackage{hyperref}
\hypersetup{
    colorlinks=true,            %链接颜色
    linkcolor=blue,             %内部链接
    filecolor=magenta,          %本地文档
    urlcolor=cyan,              %网址链接
    pdftitle={Overleaf Example},
    pdfpagemode=FullScreen,
}
\usepackage[none]{hyphenat}	% 阻止长单词分在两行
\usepackage{mathrsfs}	% 提供\mathscr字体

\RequirePackage[many]{tcolorbox}
\tcbset{
    boxed title style={colback=magenta},
	breakable,
	enhanced,
	sharp corners,
	attach boxed title to top left={yshift=-\tcboxedtitleheight,  yshifttext=-.75\baselineskip},
	boxed title style={boxsep=1pt,sharp corners},
    fonttitle=\bfseries\sffamily,
}

\definecolor{skyblue}{rgb}{0.54, 0.81, 0.94}

\newtcolorbox[auto counter, number within=chapter, number format=\arabic]{exercise}[1][]{
    title={Exercise~\thetcbcounter},
    colframe=skyblue,
    colback=skyblue!12!white,
    boxed title style={colback=skyblue},
    overlay unbroken and first={
        \node[below right,font=\small,color=skyblue,text width=.8\linewidth]
        at (title.north east) {#1};
    }
}

\newtcolorbox[auto counter, number within=chapter, number format=\arabic]{solution}[1][]{
%    top=2ex,
%    boxrule=0pt,
%    leftrule=1.4pt,
    title={Solution~\thetcbcounter},
    colframe=teal!60!green,
    colback=green!12!white,
    boxed title style={colback=teal!60!green},
    overlay unbroken and first={
        \node[below right,font=\small,color=red,text width=.8\linewidth]
        at (title.north east) {#1};
    }
}

\newtcolorbox{remark}[1][]{
    title={Remark},
    colframe=yellow!45!orange,
    colback=yellow!45!white,
    coltitle=white,
    boxed title style={colback=yellow!45!orange},
    overlay unbroken and first={
        \node[below right,font=\small,color=white,text width=.8\linewidth]
        at (title.north east) {#1};
    }
}

\begin{document}

	\setcounter{chapter}{8}

	\begin{exercise}
		To what irreducible representations can the following direct product representations be reduced for the specified point group?
		\begin{enumerate}[label=(\alph*)]
		\item $\Gamma^{A_1}\otimes\Gamma^{A_1}$, $\Gamma^{A_1}\otimes\Gamma^{A_2}$, $\Gamma^{A_2}\otimes\Gamma^{E}$, $\Gamma^{E}\otimes\Gamma^{E}$ for $\mathscr{C}_{\rm 3v}$
		\item $\Gamma^{E^\prime}\otimes\Gamma^{E^\prime}$, $\Gamma^{A^{\prime\prime}_1}\otimes\Gamma^{A^{\prime\prime}_2}$, $\Gamma^{A^{\prime\prime}_2}\otimes\Gamma^{E^{\prime\prime}}$ for $\mathscr{D}_{\rm 3h}$
		\item $\Gamma^{E_1}\otimes\Gamma^{E_1}$, $\Gamma^{E_1}\otimes\Gamma^{E_2}$, $\Gamma^{E_2}\otimes\Gamma^{E_2}$ for $\mathscr{C}_{\rm 5v}$.
		\end{enumerate}
	\end{exercise}

	\begin{solution}

		There are two methods. I will apply one for the first issue and the other for the second and third issue.

		\begin{enumerate}[label=(\alph*)]
	
		\item Firstly, we show the character table of the point group $\mathscr{C}_{\rm 3v}$.
		\begin{center}
		\begin{tabular}{cccc}\hline
	$\mathscr{C}_{\rm 3v}$ & $E$ & $2C_3$ & $3\sigma_v$ \\ \hline
			$A_1$	&	1	&	1	&	1	\\
			$A_2$	&	1	&	1	&	-1	\\
			$E$		&	2	&	-1	&	0	\\ \hline
		\end{tabular}
		\end{center}
		
		With (8-3.6) and (8-3.10), if we assume
		\begin{equation*}
			\Gamma^{A_1} \otimes \Gamma^{A_1} = a_1 \Gamma^{A_1} \oplus a_2 \Gamma^{A_2} \oplus e \Gamma^{E},
		\end{equation*}				
		where $a_1$, $a_2$ and $e$ are variables to be solved, then for class $\{E\}$,
		\begin{equation*}
			\chi^{\Gamma^{A_1}\otimes\Gamma^{A_1}}(E) = \chi^{\Gamma^{A_1}}(E) \chi^{\Gamma^{A_1}}(E) = a_1 \chi^{A_1}(E) + a_2 \chi^{A_2}(E) + e \chi^{E}(E),
		\end{equation*}
		it will be
		\begin{equation*}
			1 \times a_1 + 1 \times a_2 + 2 \times e = 1 \times 1 = 1.
		\end{equation*}
		Similarly, for classes $\{2C_3\}$ and $\{3\sigma_v\}$, we obtain
		\begin{align*}
			a_1 + a_2 - e &= 1 , \\
			a_1 - a_2 &= 1 .
		\end{align*}
		Solve the group of linear equations in $Ax=b$ form, viz.
		\begin{equation*}
			\begin{pmatrix}
				1 & 1 & 2 \\
				1 & 1 & -1 \\
				1 & -1 & 0
			\end{pmatrix}
			\begin{pmatrix}
				a_1 \\ a_2 \\ e
			\end{pmatrix} =
			\begin{pmatrix}
				1 \\ 1 \\ 1
			\end{pmatrix},
		\end{equation*}
		it is easy to find
		\begin{equation*}
			a_1 = 1 , \quad a_2 = 0 , \quad e = 0.
		\end{equation*}
		Thus,
		\begin{equation}
			\Gamma^{A_1} \otimes \Gamma^{A_1} = 1 \times \Gamma^{A_1} \oplus 0 \times \Gamma^{A_2} \oplus 0 \times \Gamma^{E} = \Gamma^{A_1}.
		\end{equation}
		
		In the same way, what we need to change for different direct products is the vector $b$. We can obtain
		\begin{align}
			\Gamma^{A_1}\otimes\Gamma^{A_2} &= \Gamma^{A_2}, \\
			\Gamma^{A_2}\otimes\Gamma^{E} &= \Gamma^{E}, \\
			\Gamma^{E}\otimes\Gamma^{E} &= \Gamma^{A_1} \oplus \Gamma^{A_2} \oplus \Gamma^{E}.
		\end{align}
		
		\item Firstly, the character table of the point group $\mathscr{D}_{\rm 3h}$ should be demonstrated.
		\begin{center}
		\begin{tabular}{ccccccc}\hline
	$\mathscr{D}_{\rm 3h}$ & $E$ & $2C_3$ & $3C^\prime_2$ & $\sigma_h$ & $2S_3$ &$3\sigma_v$ \\ \hline
			$A^\prime_1$	&	1	&	1	&	1	&	1	&	1	&	1	\\
			$A^\prime_2$	&	1	&	1	&	-1	&	1	&	1	&	-1	\\
			$E^\prime$		&	2	&	-1	&	0	&	2	&	-1	&	0	\\
			$A^{\prime\prime}_1$	&	1	&	1	&	1	&	-1	&	-1	&	-1	\\
			$A^{\prime\prime}_2$	&	1	&	1	&	-1	&	-1	&	-1	&	1	\\
			$E^{\prime\prime}$	&	2	&	-1	&	0	&	-2	&	1	&	0	\\ \hline
		\end{tabular}
		\end{center}
		
		We can calculate reduction coefficients of direct product $\Gamma^{E^\prime}\otimes\Gamma^{E^\prime}$ via (8-3.11). For instance, for the irreducible representation $A^\prime_1$,
		\begin{align*}
			a^\prime_{1} &= \frac{1}{12} ( 1 \times 2 \times 2 \times 1 + 2 \times (-1) \times (-1) \times 1 + 3 \times 0 \times 0 \times 1 \\
			&\hspace{4em} + 1 \times 2 \times 2 \times 1 + 2 \times (-1) \times (-1) \times 1 + 3 \times 0 \times 0 \times 1 ) \\
			&= \frac{1}{12} \times ( 4 + 2 + 0 + 4 + 2 + 0 ) = 1.
		\end{align*}
		In the same way, we can calculate others' reduction coefficients.
		\begin{align*}
			a^\prime_{2} &= \frac{1}{12} ( 4 + 2 - 0 + 4 + 2 - 0 )= 1, \\
			e^\prime &= \frac{1}{12} ( 8 - 2 + 0 + 8 - 2 + 0 )= 1, \\
			a^{\prime\prime}_{1} &= \frac{1}{12} ( 4 + 2 + 0 - 4 - 2 - 0 )= 0, \\
			a^{\prime\prime}_{2} &= \frac{1}{12} ( 4 + 2 - 0 - 4 - 2 + 0 )= 0, \\
			e^{\prime\prime} &= \frac{1}{12} ( 8 - 2 + 0 - 8 + 2 + 0 )= 0. 
		\end{align*}
		Finally, 
		\begin{equation}
			\Gamma^{E^\prime}\otimes\Gamma^{E^\prime} = \Gamma^{A^\prime_1} \oplus \Gamma^{A^\prime_2} \oplus \Gamma^{E^\prime}.
		\end{equation}
		Similarly,
		\begin{align}
			\Gamma^{A^{\prime\prime}_1}\otimes \Gamma^{A^{\prime\prime}_2} &= \Gamma^{A^{\prime}_2}, \\
			\Gamma^{A^{\prime\prime}_2}\otimes \Gamma^{E^{\prime\prime}} &= \Gamma^{E^{\prime}}.
		\end{align}
		
		\item Firstly, the character table of the point group $\mathscr{C}_{\rm 5v}$ should be shown.
		\begin{center}
		\begin{tabular}{ccccc}\hline
	$\mathscr{C}_{\rm 5v}$ & $E$ & $2C_5$ & $2C^2_5$ & $5\sigma_v$ \\ \hline
			$A_1$	&	1	&	1	&	1	&	1	\\
			$A_2$	&	1	&	1	&	1	&	-1	\\
			$E_1$	&	2	&	2$\cos{\frac{2\pi}{5}}$	&	2$\cos{\frac{4\pi}{5}}$	&	0	\\
			$E_2$	&	2	&	2$\cos{\frac{4\pi}{5}}$	&	2$\cos{\frac{2\pi}{5}}$	&	0	\\ \hline
		\end{tabular}
		\end{center}
		
		Here, we should note
		\begin{align*}
			\cos{36^\circ} &= \cos{\frac{\pi}{5}} = \frac{\sqrt{5}+1}{4},	\\
			\cos{72^\circ} &=\cos{\frac{2\pi}{5}} = \frac{\sqrt{5}-1}{4}.
\end{align*}				
		
		The similar calculation process is omitted. The final result is
		\begin{align}
			\Gamma^{E_1}\otimes\Gamma^{E_1} &= \Gamma^{A_1}\oplus \Gamma^{A_2} \oplus \Gamma^{E_2}, \\
			\Gamma^{E_1}\otimes\Gamma^{E_2} &= \Gamma^{E_1}\oplus \Gamma^{E_2}, \\
			\Gamma^{E_2}\otimes\Gamma^{E_2} &= \Gamma^{A_1}\oplus \Gamma^{A_2} \oplus \Gamma^{E_1}.
		\end{align}
		
		\end{enumerate}				
		
	\end{solution}
	
	\begin{remark}
		In fact, this exercise is to calculate the multiplication table of a given point group. These results can be found in many handbooks or websites in this field. Two useful websites are listed below.
		\begin{itemize}
		
		\item \url{https://zh.webqc.org/symmetry.php}
	
		\item \url{http://symmetry.jacobs-university.de}

		\end{itemize}			
		 
	\end{remark}

	\begin{exercise}
		To what irreducible representation must $\psi^\sigma$ belong if the integral
		\begin{equation*}
			\int \psi^\sigma(X)^* F^\lambda(X) \psi^\rho (X) {\rm d}\tau
		\end{equation*}
		is to be non-zero in the following cases?
		\begin{enumerate}[label=(\alph*)]
		\item $\mathscr{C}_{\rm 4v}$  $\Gamma^\lambda = \Gamma^E$; $\Gamma^\rho = \Gamma^{A_1}$, $\Gamma^{A_2}$, $\Gamma^{B_1}$, $\Gamma^{B_2}$
		\item $\mathscr{D}_{\rm 6h}$  $\Gamma^\lambda = \Gamma^{E_{1u}}$; $\Gamma^\rho = \Gamma^{E_{2u}}$
		\item $\mathscr{T}_{\rm d}$  $\Gamma^\lambda = \Gamma^{T_2}$; $\Gamma^\rho = \Gamma^{A_2}$, $\Gamma^{E}$, $\Gamma^{T_1}$, $\Gamma^{T_2}$.
		\end{enumerate}
		
	\end{exercise}
	
	\begin{solution}
	
		We should reduce these direct products to find which irreducible representations are included in them.
		
		\begin{enumerate}[label=(\alph*)]
		
		\item The character table of the point group $\mathscr{C}_{\rm 4v}$ is shown below.
		\begin{center}
		\begin{tabular}{cccccc}\hline
	$\mathscr{C}_{\rm 4v}$ & $E$ & $2C_4$ & $C_2$ & $2\sigma_v$ & $2\sigma_d$ \\ \hline
			$A_1$	&	1	&	1	&	1	&	1	&	1	\\
			$A_2$	&	1	&	1	&	1	&	-1	&	-1	\\
			$B_1$	&	1	&	-1	&	1	&	1	&	-1	\\
			$B_2$	&	1	&	-1	&	1	&	-1	&	1	\\
			$E$		&	2	&	0	&	-2	&	0	&	0\\ \hline
		\end{tabular}
		\end{center}
		These results can be calculated,
		\begin{align}
			\Gamma^{E} \otimes \Gamma^{A_1} &= \Gamma^E , \\
			\Gamma^{E} \otimes \Gamma^{A_2} &= \Gamma^E , \\
			\Gamma^{E} \otimes \Gamma^{B_1} &= \Gamma^E , \\
			\Gamma^{E} \otimes \Gamma^{B_2} &= \Gamma^E .
		\end{align}
		
		Finally, we conclude that only $\psi^\sigma$ will belong to $\Gamma^E$ to obtain a non-zero integral if $\Gamma^\lambda = \Gamma^E$ and $\Gamma^\rho = \Gamma^{A_1}$. The same conclusion also applies to cases where $\Gamma^\rho$ equals to $\Gamma^{A_2}$, $\Gamma^{B_1}$, or $\Gamma^{B_2}$.
		
		\item The character table of the point group $\mathscr{D}_{\rm 6h}$ is shown below.
		\begin{center}
		\begin{tabular}{ccccccccccccc}\hline
	$\mathscr{D}_{\rm 6h}$ & $E$ & $2C_6$ & $2C_3$ & $C_2$ & $3C^\prime_2 $ & $3C^{\prime\prime}_2$ & $i$ & $2S_3$ & $2S_6$ & $\sigma_h$ & $3\sigma_d$ & $3\sigma_v$ \\ \hline
			$A_{1g}$ & 1 & 1 & 1 & 1 & 1 & 1 & 1 & 1 & 1 & 1 & 1 & 1 \\
			$A_{2g}$ & 1 & 1 & 1 & 1 & -1 & -1 & 1 & 1 & 1 & 1 & -1 & -1 \\
			$B_{1g}$ & 1 & -1 & 1 & -1 & 1 & -1 & 1 & -1 & 1 & -1 & 1 & -1 \\
			$B_{2g}$ & 1 & -1 & 1 & -1 & -1 & 1 & 1 & -1 & 1 & -1 & -1 & 1 \\
			$E_{1g}$ & 2 & 1 & -1 & -2 & 0 & 0 & 2 & 1 & -1 & -2 & 0 & 0 \\
			$E_{2g}$ & 2 & -1 & -1 & 2 & 0 & 0 & 2 & -1 & -1 & 2 & 0 & 0 \\
			$A_{1u}$ & 1 & 1 & 1 & 1 & 1 & 1 & -1 & -1 & -1 & -1 & -1 & -1 \\
			$A_{2u}$ & 1 & 1 & 1 & 1 & -1 & -1 & -1 & -1 & -1 & -1 & 1 & 1 \\
			$B_{1u}$ & 1 & -1 & 1 & -1 & 1 & -1 & -1 & 1 & -1 & 1 & -1 & 1 \\
			$B_{2u}$ & 1 & -1 & 1 & -1 & -1 & 1 & -1 & 1 & -1 & 1 & 1 & -1 \\
			$E_{1u}$ & 2 & 1 & -1 & -2 & 0 & 0 & -2 & -1 & 1 & 2 & 0 & 0 \\
			$E_{2u}$ & 2 & -1 & -1 & 2 & 0 & 0 & -2 & 1 & 1 & -2 & 0 & 0 \\ \hline
		\end{tabular}
		\end{center}
		Thus, in the same way,
		\begin{align}
			\Gamma^{E_{1u}} \otimes \Gamma^{E_{2u}} &= \Gamma^{B_{1g}} \oplus \Gamma^{B_{2g}} \oplus \Gamma^{E_{1g}}.
		\end{align}
		
		Finally, we conclude that only $\psi^\sigma$ will belong to $\Gamma^{B_{1g}}$, $\Gamma^{B_{2g}}$, or $\Gamma^{E_{1g}}$ to obtain a non-zero integral if $\Gamma^\lambda = \Gamma^{E_{1u}}$ and $\Gamma^\rho = \Gamma^{E_{2u}}$.
		
		\item The character table of the point group $\mathscr{T}_{\rm d}$ is shown below.
		\begin{center}
		\begin{tabular}{cccccc}\hline
	$\mathscr{T}_{\rm d}$ & $E$ & $8C_3$ & $3C_2$ & $6S_4$ & $6\sigma_d$ \\ \hline
			$A_1$	&	1	&	1	&	1	&	1	&	1	\\
			$A_2$	&	1	&	1	&	1	&	-1	&	-1	\\
			$E$		&	2	&	-1	&	2	&	0	&	0	\\
			$T_1$	&	3	&	0	&	-1	&	1	&	-1	\\
			$T_2$	&	3	&	0	&	-1	&	-1	&	1\\ \hline
		\end{tabular}
		\end{center}
		Thus, in the same way,
		\begin{align}
			\Gamma^{T_2} \otimes \Gamma^{A_2} &= \Gamma^{T_1} , \\
			\Gamma^{T_2} \otimes \Gamma^{E} &= \Gamma^{T_1} \oplus \Gamma^{T_2} , \\
			\Gamma^{T_2} \otimes \Gamma^{T_1} &= \Gamma^{A_2} \oplus \Gamma^{E} \oplus \Gamma^{T_1} \oplus \Gamma^{T_2} , \\
			\Gamma^{T_2} \otimes \Gamma^{T_2} &= \Gamma^{A_1} \oplus \Gamma^{E} \oplus \Gamma^{T_1} \oplus \Gamma^{T_2} .
		\end{align}
		
		Finally, we conclude that 
			\begin{enumerate}[label=(\arabic*)]
			
			\item Only $\psi^\sigma$ will belong to $\Gamma^{T_1}$ to obtain a non-zero integral if $\Gamma^\lambda = \Gamma^{T_2}$ and $\Gamma^\rho = \Gamma^{A_2}$.
			
			\item Only $\psi^\sigma$ will belong to $\Gamma^{T_1}$, or $\Gamma^{T_2}$ to obtain a non-zero integral if $\Gamma^\lambda = \Gamma^{T_2}$ and $\Gamma^\rho = \Gamma^{E}$.
				
			\item Only $\psi^\sigma$ will belong to $\Gamma^{A_2}$, $\Gamma^E$, $\Gamma^{T_1}$, or $\Gamma^{T_2}$ to obtain a non-zero integral if $\Gamma^\lambda = \Gamma^{T_2}$ and $\Gamma^\rho = \Gamma^{T_1}$.
				
			\item Only $\psi^\sigma$ will belong to $\Gamma^{A_1}$, $\Gamma^E$, $\Gamma^{T_1}$, or $\Gamma^{T_2}$ to obtain a non-zero integral if $\Gamma^\lambda = \Gamma^{T_2}$ and $\Gamma^\rho = \Gamma^{T_2}$.
			
			\end{enumerate}					
		
		\end{enumerate}
	
	\end{solution}

\end{document}